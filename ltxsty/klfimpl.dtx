% \iffalse meta-comment
%
% Copyright (C) 2021 by Philippe Faist, philippe.faist@bluewin.ch
% -------------------------------------------------------
% 
% This file is part of the klfengine sources.  See LICENSE file
% for more information.
%
% \fi
%
% \iffalse
%<cxxdata>const char * klfimpl_sty_data =
%<cxxdata>  R"%%%###%%%###%%%(%% This is klfimpl.sty exported as C++ source code
%<*driver>
\ProvidesFile{klfimpl.dtx}
%</driver>
%<package|cxxdata>\NeedsTeXFormat{LaTeX2e}[2005/12/01]
%<package|cxxdata>\ProvidesPackage{klfimpl}
%<*package|cxxdata>
    [2020/08/20 v0.2 klfimpl package]
%</package|cxxdata>
%
%<*driver>
\documentclass{ltxdoc}
\usepackage{xcolor}
\usepackage[preset=xpkgdoc]{phfnote}
\EnableCrossrefs         
\CodelineIndex
\RecordChanges
\begin{document}
  \DocInput{klfimpl.dtx}
\end{document}
%</driver>
% \fi
%
% \CheckSum{0}
%
% \CharacterTable
%  {Upper-case    \A\B\C\D\E\F\G\H\I\J\K\L\M\N\O\P\Q\R\S\T\U\V\W\X\Y\Z
%   Lower-case    \a\b\c\d\e\f\g\h\i\j\k\l\m\n\o\p\q\r\s\t\u\v\w\x\y\z
%   Digits        \0\1\2\3\4\5\6\7\8\9
%   Exclamation   \!     Double quote  \"     Hash (number) \#
%   Dollar        \$     Percent       \%     Ampersand     \&
%   Acute accent  \'     Left paren    \(     Right paren   \)
%   Asterisk      \*     Plus          \+     Comma         \,
%   Minus         \-     Point         \.     Solidus       \/
%   Colon         \:     Semicolon     \;     Less than     \<
%   Equals        \=     Greater than  \>     Question mark \?
%   Commercial at \@     Left bracket  \[     Backslash     \\
%   Right bracket \]     Circumflex    \^     Underscore    \_
%   Grave accent  \`     Left brace    \{     Vertical bar  \|
%   Right brace   \}     Tilde         \~}
%
%
% \GetFileInfo{klfimpl.dtx}
%
% \DoNotIndex{\newcommand,\newenvironment,\def,\gdef,\edef,\xdef,\if,\else,\fi,\ifx}
% 
% \title{The \pkgname{klfimpl} package\thanks{\itshape This document corresponds
% to \pkgname{klfimpl}~\fileversion, dated \filedate. It is part of the
% klfengine tool, see \url{https://github.com/klatexformula/klfengine}.}}
% \author{Philippe Faist\quad\email{philippe.faist@bluewin.ch}}
% \date{\pkgfmtdate\filedate}
% \maketitle
%
% \begin{abstract}
%   \pkgname{klfimpl}---LaTeX helper code for the implementation of
%   the klfengine (the engine of KLatexFormula~5).
% \end{abstract}
%
% \inlinetoc
%
% \section{Introduction}
%
% \emph{KLatexFormula~5} will ship with a new engine, \emph{klfengine}.  This
% implementation converts a LaTeX equation to different formats such as PDF or
% PNG.  While KLatexFormula up to version 4 required a few runs of |gs| and
% other manual EPS fixes after running |latex| and |dvips|, the new workflow
% aims at directly doing as muh as possible from the latex end of things---get
% the correct paper size, correct background color, page margins, and so
% on---therefore cleaning up the workflow and speeding up the whole compilation.
%
% This package provides an environment |\begin{klfcontent}...\end{klfcontent}|
% which typesets the given content in a box, measures the box dimensions, then
% resizes the page to fit the box as requested including margins and/or
% alignment to some fixed size, and renders the whole thing.
%
% This package expects that there is a single such environment in the entire
% document and no other content.
%
%
%
% \StopEventually{\PrintChanges\PrintIndex}
%
% \section{Implementation}
%
% \subsection{Some general declarations}
%
% The box in which the LaTeX content will be typeset.
%    \begin{macrocode}
\newbox\klf@eqnbox
%    \end{macrocode}
%
% Some dimensions etc. that we will track.  The main box dimensions (width,
% height, depth, total height = height + depth):
%    \begin{macrocode}
\newdimen\klf@w
\newdimen\klf@h
\newdimen\klf@d
\newdimen\klf@th
%    \end{macrocode}
% The paper size (width/height):
%    \begin{macrocode}
\newdimen\klf@ppw
\newdimen\klf@pph
%    \end{macrocode}
% Any offset that should be used to display the box (useful if margins are
% requested, or a fixed page width and/or height are requested):
%    \begin{macrocode}
\newdimen\klf@hshift
\newdimen\klf@vshift
%    \end{macrocode}
% And record some font dimensions:
%    \begin{macrocode}
\newdimen\klf@em
\newdimen\klf@ex
\newdimen\klf@capxhgt
%    \end{macrocode}
%
% We will compute any user input dimensions as proper dimension \emph{inside}
% the equation box to make sure font settings are taken into account correctly.
% The dimensions will be stored here.
%    \begin{macrocode}
\newdimen\klf@dim@fixedwidth
\newdimen\klf@dim@fixedheight
\newdimen\klf@dim@topmargin
\newdimen\klf@dim@rightmargin
\newdimen\klf@dim@bottommargin
\newdimen\klf@dim@leftmargin
%    \end{macrocode}
%
% Remember which engine we're running under.  To speed up things and to avoid
% problems with like |iftex| package not existing, we simply require the caller
% to tell us via a package option what |latex| engine is being run (|latex| with
% DVI output, |pdflatex|, |xelatex|, or |lualatex|).
%    \begin{macrocode}
\newif\ifklf@ltxengine@latex \klf@ltxengine@latexfalse
\newif\ifklf@ltxengine@pdflatex \klf@ltxengine@pdflatexfalse
\newif\ifklf@ltxengine@xelatex \klf@ltxengine@xelatexfalse
\newif\ifklf@ltxengine@lualatex \klf@ltxengine@lualatexfalse
%    \end{macrocode}
%
% If, for some reason, this package is called in a different context where we
% wouldn't want the layout to be set to zero by default, then there is a package
% option for this (|keeplayoutsizes|).  This is the corresponding |\newif| flag:
%    \begin{macrocode}
\newif\ifklf@keeplayoutsizes
\klf@keeplayoutsizesfalse
%    \end{macrocode}
%
% The user can specify a fixed width and/or a fixed height for the resulting
% layout.  These will be stored here, if applicable (or they will remain empty).
% The user input should be stored as a \emph{macro}, not as a \emph{dimen},
% because we want a dimension given in font-specific metrics (e.g. |4.2em|) to
% be computed correctly relative to the equation font.
%    \begin{macrocode}
\def\klf@set@fixedwidth{}
\def\klf@set@fixedheight{}
%    \end{macrocode}
% Enable user to specify margins around the equations.
%    \begin{macrocode}
\def\klf@set@topmargin{0.1ex}
\def\klf@set@rightmargin{0.1ex}
\def\klf@set@bottommargin{0.1ex}
\def\klf@set@leftmargin{0.1ex}
%    \end{macrocode}
% Enable user to specify global scaling factors for the full resulting box.
% Scale will be applied to margins as well as to fixed size.
%    \begin{macrocode}
\def\klf@set@xscale{1}
\def\klf@set@yscale{1}
%    \end{macrocode}
% User can specify how the equation is aligned inside the box, in case we have a
% fixed width or a fixed height.  The horizontal (resp.\@ vectical) alignment
% coefficient is |0| for left (resp.\@ top) alignment, |0.5| for middle
% alignment, and |1| for right (resp.\@ bottom) alignment.  It can be any value
% that interpolates between these.
%    \begin{macrocode}
\def\klf@set@xaligncoeff{0.5}
\def\klf@set@yaligncoeff{0.5}
%    \end{macrocode}
% User can specify the reference points for top and bottom alignment.  The top
% can be aligned to the natural height of the box (|bbox|, the default) or to
% the height of a capital 'X' (|Xheight|).  The bottom can be aligned to the
% natural depth of the box (|bbox|, the default) or to the baseline
% (|baseline|).  (When you set |baseline|, make sure you have a fixed size or
% reasonable margins to accomodate any box depth.)
%    \begin{macrocode}
\def\klf@set@topalignment{bbox}
\def\klf@set@bottomalignment{bbox}
%    \end{macrocode}
% User can tell if they would like a line to be drawn where the baseline is, for
% instance as a visual reference for use in vector graphics editing software.
% This can be |none| or |line|.
%    \begin{macrocode}
\def\klf@set@baselineruletype{none}
%    \end{macrocode}
% 
% 
% \subsection{Package options and settings}
%
% Set the \LaTeX\space engine.
%    \begin{macrocode}
\DeclareOption{latex}{\klf@ltxengine@latextrue}
\DeclareOption{pdflatex}{\klf@ltxengine@pdflatextrue}
\DeclareOption{xelatex}{\klf@ltxengine@xelatextrue}
\DeclareOption{lualatex}{\klf@ltxengine@lualatextrue}
%    \end{macrocode}
%
% Package option to inhibit resetting the page layout to zero by default.
%    \begin{macrocode}
\DeclareOption{keeplayoutsizes}{\klf@keeplayoutsizestrue}
%    \end{macrocode}
% 
% 
% Now process those options
%    \begin{macrocode}
\DeclareOption*{\PackageError{klfimpl}{Unknown option ‘\CurrentOption’}{}}
\ProcessOptions\relax
%    \end{macrocode}
%
% The following could have been specified as package options, but for the sake
% of simplicity (and to avoid having to use \pkgname{keyval}/\pkgname{xkeyval}
% and so on, the information is provided via a simple macro call.
%
% \begin{macro}{\klfSetFixedWidth}
% \begin{macro}{\klfSetFixedHeight}
% If applicable, set the fixed width and/or fixed height of the content to
% typeset.
%    \begin{macrocode}
\def\klfSetFixedWidth#1{%
  \xdef\klf@set@fixedwidth{#1}}
\def\klfSetFixedHeight#1{%
  \xdef\klf@set@fixedheight{#1}}
%    \end{macrocode}
% \end{macro}
% \end{macro}
% 
% \begin{macro}{\klfSetTopMargin}
% \begin{macro}{\klfSetRightMargin}
% \begin{macro}{\klfSetBottomMargin}
% \begin{macro}{\klfSetLeftMargin}
% Same for equation margins:
%    \begin{macrocode}
\def\klfSetTopMargin#1{%
  \xdef\klf@set@topmargin{#1}}
\def\klfSetRightMargin#1{%
  \xdef\klf@set@rightmargin{#1}}
\def\klfSetBottomMargin#1{%
  \xdef\klf@set@bottommargin{#1}}
\def\klfSetLeftMargin#1{%
  \xdef\klf@set@leftmargin{#1}}
%    \end{macrocode}
% \end{macro}
% \end{macro}
% \end{macro}
% \end{macro}
%
% \begin{macro}{\klfSetXScale}
% \begin{macro}{\klfSetYScale}
% \begin{macro}{\klfSetScale}
%   Specify horizontal and vectical scaling factors.  Values are a multipliying
%   factor for dimensions: the value |1| means original size, |0.5| half size,
%   |2| double size.  If a value different than |1| was specified, the
%   \pkgname{graphics} package is loaded.  {\itshape NOTE: This has to be the
%   exact expression |1|, not |1.0| or |1.|.}  The |\klfSetScale| macro is a
%   convenience macro that sets both horizontal and vectical scaling factors to
%   the same value.
%    \begin{macrocode}
\def\klfSetXScale#1{%
  \xdef\klf@set@xscale{#1}%
  \ifx\klf@set@xscale\klf@macro@one
  \else
    \RequirePackage{graphics}%
  \fi
}
\def\klfSetYScale#1{%
  \xdef\klf@set@yscale{#1}%
  \ifx\klf@set@yscale\klf@macro@one
  \else
    \RequirePackage{graphics}%
  \fi
}
\def\klfSetScale#1{%
  \xdef\klf@set@xscale{#1}%
  \xdef\klf@set@yscale{#1}%
  \ifx\klf@set@xscale\klf@macro@one
  \else
    \RequirePackage{graphics}%
  \fi
}
\def\klf@macro@one{1}
%    \end{macrocode}
% \end{macro}
% \end{macro}
% \end{macro}
%
% \begin{macro}{\klfSetXAlignCoeff}
% \begin{macro}{\klfSetYAlignCoeff}
%    \begin{macrocode}
\def\klfSetXAlignCoeff#1{%
  \xdef\klf@set@xaligncoeff{#1}%
}
\def\klfSetYAlignCoeff#1{%
  \xdef\klf@set@yaligncoeff{#1}%
}
%    \end{macrocode}
% \end{macro}
% \end{macro}
%
% \begin{macro}{\klfSetTopAlignment}
% \begin{macro}{\klfSetBottomAlignment}
%   Macros for the user to set top and bottom alignment.  See earlier for
%   explanation of possible values.  The possible values will be used to invoke
%   a macro named e.g. |\klf@correctboxheight@@|\meta{top-alignment-name} and
%   |\klf@correctboxdepth@@|\meta{bottom-alignment-name}, see below in
%   |{klfcontent}| environment.
%    \begin{macrocode}
\def\klfSetTopAlignment#1{%
  \xdef\klf@set@topalignment{#1}}
\def\klfSetBottomAlignment#1{%
  \xdef\klf@set@bottomalignment{#1}}
%    \end{macrocode}
% \end{macro}
% \end{macro}
%
% \begin{macro}{\klfSetBaselineRuleType}
%   Set which kind of baseline rule we would like, if any.
%    \begin{macrocode}
\def\klfSetBaselineRuleType#1{%
  \xdef\klf@set@baselineruletype{#1}}
%    \end{macrocode}
% 
% \subsection{Basic/common implementation macros}
% 
% First of all, a simple macro to reset all LaTeX layout dimensions.
%    \begin{macrocode}
\def\klf@ZeroLayoutSizes{%
  \oddsidemargin=\z@\relax
  \evensidemargin=\z@\relax
  \topmargin=\z@\relax
  \voffset=-1in\relax
  \hoffset=-1in\relax
  \headsep=\z@\relax
  \headheight=\z@\relax
  \marginparsep=\z@\relax
  \footskip=\z@\relax
  \parindent=\z@\relax
  \parskip=\z@\relax
  \topskip=\z@\relax
}
%    \end{macrocode}
% \begin{macro}{\klf@ZeroDisplaySkips}
% And define a routine that sets all the display-related skips to zero so that
% we can use this inside a |\vbox|.
%    \begin{macrocode}
\def\klf@ZeroDisplaySkips{%
  \abovedisplayskip=\z@\relax
  \belowdisplayskip=\z@\relax
  \abovedisplayshortskip=\z@\relax
  \belowdisplayshortskip=\z@\relax
}
%    \end{macrocode}
% \end{macro}
%
% By default, reset all these dimensions right away, unless the
% |keeplayoutsizes| package option was provided.
%    \begin{macrocode}
\ifklf@keeplayoutsizes
\else
  \klf@ZeroLayoutSizes
  \klf@ZeroDisplaySkips
\fi
%    \end{macrocode}
% 
% \begin{macro}{\klfSetPaperSize}
%   Change the paper size.  For |pdflatex| and |xe/luatex| this can be called
%   after |\begin||{document}| but for |latex| with traditional |dvi| output
%   this must be issued in the preamble.
%    \begin{macrocode}
\def\klfSetPaperSize#1#2{%
  \@tempdima=#1\relax
  \@tempdimb=#2\relax
  \klf@SetPaperSize@FromDims\@tempdima\@tempdimb
}
\def\klf@SetPaperSize@FromDims#1#2{%
  \global\textwidth=#1\relax
  \global\textheight=#2\relax
  \global\hsize=#1\relax
  \global\vsize=#2\relax
  \global\paperwidth=#1\relax
  \global\paperheight=#2\relax
  \ifklf@ltxengine@pdflatex
    \global\pdfpagewidth=#1\relax
    \global\pdfpageheight=#2\relax
  \fi
  \ifklf@ltxengine@xelatex
    \global\pdfpagewidth=#1\relax
    \global\pdfpageheight=#2\relax
  \fi
  \ifklf@ltxengine@lualatex
    \global\pagewidth=#1\relax
    \global\pageheight=#2\relax
  \fi
}
%    \end{macrocode}
% \end{macro}
%
%
% \subsection{Main implementation routine}
% 
% \begin{environment}{klfcontent}
%   The argument should be a box command (e.g., |\hbox|, |\vbox|, |\vtop|,
%   |\vcenter|).  Example usage:
%   |\begin||{klfcontent}{\vcenter}{|\meta{initialization code}|}...| or
%     |\begin||{klfcontent}{\hbox to 1cm}{|\meta{init. code}|}...|.
%    \begin{macrocode}
\def\klfcontent#1#2{%
  \unskip
  \samepage
  \setbox\klf@eqnbox=#1\bgroup
    \klf@ZeroDisplaySkips%
    #2%
    \global\klf@em=1em\relax
    \global\klf@ex=1ex\relax
    \setbox0=\hbox{X}%
    \global\klf@capxhgt=\ht0%
    \ifx\klf@set@fixedwidth\@empty\else
      \global\klf@dim@fixedwidth=\klf@set@fixedwidth\relax
    \fi
    \ifx\klf@set@fixedheight\@empty\else
      \global\klf@dim@fixedheight=\klf@set@fixedheight\relax
    \fi
    \global\klf@dim@topmargin=\klf@set@topmargin\relax
    \global\klf@dim@rightmargin=\klf@set@rightmargin\relax
    \global\klf@dim@bottommargin=\klf@set@bottommargin\relax
    \global\klf@dim@leftmargin=\klf@set@leftmargin\relax
}
\def\endklfcontent{%
  \egroup
%    \end{macrocode}
% 
% Now we record the box dimensions.
%
%    \begin{macrocode}
  \klf@w=\wd\klf@eqnbox\relax
  \klf@h=\ht\klf@eqnbox\relax
  \klf@d=\dp\klf@eqnbox\relax
%    \end{macrocode}
% If we shouldn't align to the bounding box, correct the height and the depth of
% the box to whatever is requested by the corresponding top/bottom alignment
% options.
%    \begin{macrocode}
  \csname klf@correctboxheight@@\klf@set@topalignment\endcsname
  \csname klf@correctboxdepth@@\klf@set@bottomalignment\endcsname
%    \end{macrocode}
% \iffalse
%  % Add an epsilon in each dimension to prevent rounding errors from causing
%  % unwanted breaks.  Compute the ``total height'' = height + depth.
%  %    \begin{macrocode}
%  %\advance \klf@w 0.00001pt \relax
%  %\advance \klf@h 0.00001pt \relax
% \fi
%    \begin{macrocode}
  \klf@th=\klf@h\relax
  \advance \klf@th \klf@d \relax
%    \end{macrocode}
% \emph{Determine page size and offsets.}  Take into account any possible fixed
% paper width or height and any margins.
%    \begin{macrocode}
  \ifx\klf@set@fixedwidth\@empty%
    \klf@ppw=\klf@w\relax
    \advance \klf@ppw \klf@dim@leftmargin \relax
    \advance \klf@ppw \klf@dim@rightmargin \relax
    \klf@hshift=\klf@dim@leftmargin\relax
  \else%
    \klf@ppw=\klf@dim@fixedwidth\relax
    \klf@hshift=\klf@set@xaligncoeff\klf@ppw\relax
    \advance \klf@hshift -\klf@set@xaligncoeff\klf@w\relax
    \advance \klf@hshift -\klf@set@xaligncoeff\klf@dim@rightmargin\relax
    \advance \klf@hshift -\klf@set@xaligncoeff\klf@dim@leftmargin\relax
    \advance \klf@hshift \klf@dim@leftmargin\relax
  \fi
  \ifx\klf@set@fixedheight\@empty%
    \klf@pph=\klf@th\relax
    \advance \klf@pph \klf@dim@topmargin \relax
    \advance \klf@pph \klf@dim@bottommargin \relax
    \klf@vshift=\klf@dim@topmargin\relax
  \else%
    \klf@pph=\klf@dim@fixedheight\relax
    \klf@vshift=\klf@set@yaligncoeff\klf@pph\relax
    \advance \klf@vshift -\klf@set@yaligncoeff\klf@th\relax
    \advance \klf@vshift -\klf@set@yaligncoeff\klf@dim@bottommargin\relax
    \advance \klf@vshift -\klf@set@yaligncoeff\klf@dim@topmargin\relax
    \advance \klf@vshift \klf@dim@topmargin\relax
  \fi
%    \end{macrocode}
% No scale has been applied yet.  Call the rendering routine that will take into
% account scaling factors as necessary.
%    \begin{macrocode}
  \klf@RenderContentBox
%    \end{macrocode}
% Finally dump all meta-info to the standard output to provide additional
% information back to \emph{klatexformula}.
%    \begin{macrocode}
  \klfDumpMetaInfo
  \ignorespaces
}
%    \end{macrocode}
%
% \begin{macro}{\klf@RenderContentBox}
%   The |\klf@RenderContentBox| macro sets the paper size (if the current latex
%   engine allows this at this point), and displays the box accordingly.
%
%   If we're using a pdf-based engine (|pdflatex|, |xelatex| or |lualatex|),
%   then we set the page size immediately (scaled correctly).  If we're using
%   the |latex| (\LaTeX\space $\to$ DVI) engine, then we cannot set the page
%   size at this point, because the page size needs to be set in the preamble.
%   In this case, we don't do anything now, but \emph{klatexformula} performs a
%   second pass where the correct exact page size is set in preamble (via
%   meta-data dumped by |\klfDumpMetaInfo|).
%
%    \begin{macrocode}
\def\klf@RenderContentBox{%
  \ifklf@ltxengine@latex% tough luck
  \else
    \@tempdima=\klf@ppw
    \@tempdima=\klf@set@xscale\@tempdima\relax
    \@tempdimb=\klf@pph
    \@tempdimb=\klf@set@yscale\@tempdimb\relax
    \klf@SetPaperSize@FromDims\@tempdima\@tempdimb%
  \fi
%    \end{macrocode}
%   To render the contents, we check whether any scaling is applied.  If so, we
%   call |\klf@do@scale| which wraps the argument in an appropriate |\scalebox|
%   (provided by the \pkgname{graphics} package). (Otherwise we simply render
%   the box without any |\scalebox|.)
%    \begin{macrocode}
  \let\klf@next\@firstofone
  \ifx\klf@set@xscale\klf@macro@one\else
    \let\klf@next\klf@do@scale\fi
  \ifx\klf@set@yscale\klf@macro@one\else
    \let\klf@next\klf@do@scale\fi
  \nobreak
  \klf@next{%
    \hbox to \z@{\vbox to \z@{%
        \hrule \@height\z@\nobreak
        \vskip \klf@vshift\relax\nobreak
        \hbox{\vrule \@width\z@ \relax
          \raise \klf@d \hbox to \z@{%
            \csname klf@baseline@rule@@\klf@set@baselineruletype\endcsname
          }%
          \hskip \klf@hshift\relax
          \raise \klf@d \box\klf@eqnbox}%
      }}%
  }%
}
\def\klf@do@scale#1{%
  \scalebox{\klf@set@xscale}[\klf@set@yscale]{#1}%
}
%    \end{macrocode}
% \end{macro}
% \end{environment}
% 
% Now we define the top/bottom alignment correction routines.  These ``fix'' the
% height and the depth of the box (rather, their values recorded in |\klf@h| and
% |\klf@d|) according to the given options.
%
% The |bbox| top and bottom alignment options is the default, and leaves the box
% dimensions unchanged.
%    \begin{macrocode}
\def\klf@correctboxheight@@bbox{}
\def\klf@correctboxdepth@@bbox{}
%    \end{macrocode}
% The |Xheight| top alignment option sets the height of the ``box'' to be the
% height of a capital ``X''.  WARNING: This assumes that the box only contains a
% single line of text.
%    \begin{macrocode}
\def\klf@correctboxheight@@Xheight{%
  \klf@h=\klf@capxhgt
}
%    \end{macrocode}
% The |baseline| bottom alignment option sets the bottom of the ``box'' to be
% the baseline, so sets the depth to zero.
%    \begin{macrocode}
\def\klf@correctboxdepth@@baseline{%
  \klf@d=\z@
}
%    \end{macrocode}
%
% Now we define the baseline rule types.  These simply draw whatever they want,
% typically a simple |\vrule| of a given width (because we're in horizontal
% mode).
%
% First, we have the |none| rule type which simply typesets nothing.
%    \begin{macrocode}
\def\klf@baseline@rule@@none{}
%    \end{macrocode}
% 
% Then we have the |line| rule, which draws a horizontal line throughout the box
% at the baseline height.
% \begin{macro}{\klfBaselineRuleLineSetup}
% \begin{macro}{\klfBaselineRuleLineThickness}
%   The line is controlled by the macros |\klfBaselineRuleLineSetup| and
%   |\klfBaselineRuleLineThickness|:
%    \begin{macrocode}
\def\klfBaselineRuleLineSetup{}
\def\klfBaselineRuleLineThickness{0.05\p@}
%    \end{macrocode}
% You may redefine these to style the line as appropriate.  For instance, this
% would give you a blue baseline that is 0.2pt thick:
% \begin{verbatim}
% \renewcommand\klfBaselineRuleLineSetup{\color{blue}}
% \renewcommand\klfBaselineRuleLineThickness{0.2pt}
% \end{verbatim}
%
% And finally this is the code that draws the line.
%    \begin{macrocode}
\def\klf@baseline@rule@@line{%
  \begingroup
    \klfBaselineRuleLineSetup
    \vrule width\klf@ppw height\z@ depth\klfBaselineRuleLineThickness\relax
  \endgroup
}
%    \end{macrocode}
% \end{macro}
% \end{macro}
% \end{macro}
%
% \subsection{Dump meta-info on standard output}
%
% Define the routine that communicates back to \emph{klatexformula} meta-info
% about the typeset content.  This is automatically called in
% |\end||{klfcontent}|.
%
% \begin{macro}{\klfDumpMetaInfo}
%   Dump meta-info on standard output to provide additional information to
%   KLatexFormula.  Careful, scaling factors have not been applied yet.  So
%   apply them here before displaying the quantities.
%    \begin{macrocode}
\def\klfDumpMetaInfo{%
  \begingroup
    \klf@em=\klf@set@xscale\klf@em\relax
    \klf@ex=\klf@set@yscale\klf@ex\relax
    \klf@capxhgt=\klf@set@yscale\klf@capxhgt\relax
    \klf@ppw=\klf@set@xscale\klf@ppw\relax
    \klf@pph=\klf@set@yscale\klf@pph\relax
    \klf@hshift=\klf@set@xscale\klf@hshift\relax
    \klf@vshift=\klf@set@yscale\klf@vshift\relax
    \klf@w=\klf@set@xscale\klf@w\relax
    \klf@h=\klf@set@yscale\klf@h\relax
    \klf@d=\klf@set@yscale\klf@d\relax
    \klf@th=\klf@set@yscale\klf@th\relax
    \message{%
^^J%
***-KLF-META-INFO-BEGIN-***^^J%
EM={\the\klf@em}^^J%
EX={\the\klf@ex}^^J%
CAP_X_HEIGHT={\the\klf@capxhgt}^^J%
PAPER_WIDTH={\the\klf@ppw}^^J%
PAPER_HEIGHT={\the\klf@pph}^^J%
HSHIFT={\the\klf@hshift}^^J%
VSHIFT={\the\klf@vshift}^^J%
BOX_WIDTH={\the\klf@w}^^J%
BOX_HEIGHT={\the\klf@h}^^J%
BOX_DEPTH={\the\klf@d}^^J%
BOX_TOTALHEIGHT={\the\klf@th}^^J%
^^J%
***-KLF-META-INFO-END-***^^J%
    }
  \endgroup
}%
%    \end{macrocode}
% \end{macro}
% 
%\Finale
%\endinput
%<cxxdata>)%%%###%%%###%%%";
\endinput
